\documentclass[12pt,a4paper]{article}
\usepackage[T1]{fontenc}
\usepackage[left=2cm, right=2cm, top=2cm, bottom=2cm]{geometry}
\usepackage{graphicx}
\usepackage{mathtools}
\usepackage{amssymb}
\usepackage{amsthm}
\usepackage{hyperref}
\usepackage{braket}
\usepackage{setspace}

\title{On sum over states (SOS)}
\author{Pierre Beaujean}
\begin{document}
	\maketitle
	
	\onehalfspacing
	
	On the ground of perturbation theory, 
	the SOS expression of Orr and Ward \cite{orrPerturbationTheoryNonlinear1971} (see also Bishop \cite{bishopExplicitNondivergentFormulas1994}) states that any component of any nonlinear optical tensor $\chi^{(n)}(-\omega_\sigma;\omega_1,\ldots)$ (of order $n$)\footnote{I'm aware that $\chi$ is generaly used to refer to the nonlinear susceptibility, the macroscopic equivalent of what I'm discussing here, but I needed a greek letter.} is given by:\begin{align}
		&\chi^{(n)}_{ijk\ldots\,(\text{n times})}(-\omega_\sigma;\omega_1,\ldots) = \hbar^{-n+1}\sum_\mathcal{P}\sum_{a_1,a_2\ldots\,a_{n-1}} \frac{\mu^i_{0a_1}{\mu}^j_{a_1a_2}\ldots \mu^n_{a_{n-1\,0}}}{\prod_{0<i<n} (\omega_{a_i}-\omega_\sigma+\sum_{0<j<i} \omega_j)},\label{eq:1:sos}
	\end{align}
	where $i,j,k\ldots$ are the Cartesian coordinates $x, y, z$ (in the molecular frame), $\omega_1, \omega_2\ldots$, the (optical) input frequencies of the laser for the NLO process (with $\omega_\sigma = \sum_{0<i<n} \omega_i$), $\ket{a_1}, \ket{a_2}, \ldots$, the electronic  states of the system  \textbf{including the ground state} (with $\hbar\omega_{a_i}$ the excitation energy from ground state, noted  $\ket{0}$, to $\ket{a_i}$), $\mu^r_{a_ia_j} = \braket{a_i|\hat r|a_j}$ the transition dipole moment from state $a_i$ to $a_j$ (it corresponds to the dipole moment of electronic state $a_i$ when $i=j$), and $\sum_\mathcal{P}$ the sum of the different permutations over each pair $(i, \omega_\sigma),(j,\omega_1),\ldots$.
	
	In particular, one has:\begin{align*}
		\alpha_{ij}(-\omega;\omega) &= \hbar^{-1}\sum_\mathcal{P} \sum_{a_1} \frac{(ij)_{a_1}}{\omega_{a_1}-\omega},\\
		\beta_{ijk}(-\omega_\sigma;\omega_1,\omega_2) &= \hbar^{-2}\sum_\mathcal{P} \sum_{a_1,a_2} \frac{(ijk)_{a_1a_2}}{(\omega_{a_1}-\omega_\sigma)(\omega_{a_2}-\omega_\sigma+\omega_1)},\\
		\gamma_{ijkl}(-\omega_\sigma;\omega_1,\omega_2,\omega_3) &= \hbar^{-3}\sum_\mathcal{P} \sum_{a_1,a_2,a_3} \frac{(ijkl)_{a_1a_2a_3}}{(\omega_{a_1}-\omega_\sigma)(\omega_{a_2}-\omega_\sigma+\omega_1)(\omega_{a_2}-\omega_\sigma+\omega_1+\omega_2)},
	\end{align*}
	with the expression for the polarizability $\alpha = \chi^{(1)}$, the first hyperpolarizability, $\beta = \chi^{(2)}$, and the second hyperpolarizability, $\gamma = \chi^{(3)}$, and where the notation of Bishop \cite{bishopExplicitNondivergentFormulas1994} for the numerator $(ijkl)_{a_1a_2a_3} = \mu_{0a_1}^i\mu_{a_1a_2}^j\mu_{a_2a_3}^k\mu_{a_3\,0}^l$ is used. Any of these expression leads to divergences (singularities) when the denominator is zero. This is referred to as \textbf{secular divergence} if this is due to the fact that any $\ket{a_i} = \ket{0}$  (and thus $\omega_{a_i} = 0$), but it can also happen if any of the optical frequencies (or a combination of them) matches $\omega_{a_i}\neq 0$, which one generally refers to as \textbf{resonance} \cite{bishopExplicitNondivergentFormulas1994}. While the latter are inherent to perturbation theory and can be ``cured'' by introducing damping factors (although the correct way to do so is still debated \cite{campoPracticalModelFirst2012a}), the former are purely mathematical, since they can actually be avoided.
	
	\bibliographystyle{unsrt}
	\bibliography{biblio}
	
\end{document}