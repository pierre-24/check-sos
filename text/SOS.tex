\documentclass[12pt,a4paper]{article}
\usepackage[T1]{fontenc}
\usepackage[left=2cm, right=2cm, top=2cm, bottom=2cm]{geometry}
\usepackage{graphicx}
\usepackage{mathtools}
\usepackage{amssymb}
\usepackage{amsthm}
\usepackage{hyperref}
\usepackage{braket}
\usepackage{setspace}

\title{On sum over states (SOS)}
\author{Pierre Beaujean}
\begin{document}
\maketitle

\allowdisplaybreaks
\onehalfspacing

On the ground of perturbation theory, 
the SOS expression of Orr and Ward \cite{orrPerturbationTheoryNonlinear1971} (see also Bishop \cite{bishopExplicitNondivergentFormulas1994}) states that any component of any nonlinear optical tensor $\chi^{(n)}(-\omega_\sigma;\omega_1,\ldots)$ (of order $n$)\footnote{I'm aware that $\chi$ is generaly used to refer to the nonlinear susceptibility, the macroscopic equivalent of what I'm discussing here, but I needed a greek letter.} is given by:\begin{align}
&\chi^{(n)}_{\zeta\eta\ldots\nu}(-\omega_\sigma;\omega_1,\ldots) = \hbar^{-n}\sum_\mathcal{P}\sum_{a_1,a_2\ldots\,a_{n}} \frac{\mu^\zeta_{0a_1}{\mu}^\eta_{a_1a_2}\ldots \mu^\nu_{a_{n\,0}}}{\prod_{0<i\leq n} (\omega_{a_i}-\omega_\sigma+\sum_{0<j<i} \omega_j)},\label{eq:1:sos}
\end{align}
where $\zeta,\eta,\ldots$ are the Cartesian coordinates $x, y, z$ (in the molecular frame), $\omega_1, \omega_2\ldots$, the (optical) input frequencies of the laser for the NLO process (with $\omega_\sigma = \sum_{0<i<n} \omega_i$), $\ket{a_1}, \ket{a_2}, \ldots$, the states of the system  \textbf{including the ground state} (with $\hbar\omega_{a_i}$ the excitation energy from ground state, noted  $\ket{0}$, to $\ket{a_i}$), $\mu^\zeta_{a_ia_j} = \braket{a_i|\hat \zeta|a_j}$ the transition dipole moment from state $a_i$ to $a_j$ (it corresponds to the dipole moment of electronic state $a_i$ when $i=j$), and $\sum_\mathcal{P}$ the sum of the different permutations over each pair $(\zeta, \omega_\sigma),(\eta,\omega_1),\ldots$. 

Given the form of Eq.~\eqref{eq:1:sos}, it is relatively easy to write a (Python) code that compute any $\chi^{(n)}$. However, doing so requires care, since this expression blow up when any $\omega_i = 0$. The goal of this document is to find alternative formulas, while retaining generality.

\section{Theory}

\subsection{Avoiding secular divergence: using fluctuation dipole}

Examining the expressions for $n\in[1,3]$ more closely, one has:
\begin{align}
	\alpha_{ij}(-\omega;\omega) &= \hbar^{-1} \sum_\mathcal{P} \sum_{a_1} \frac{(\zeta\eta)_{a_1}}{\omega_{a_1} - \omega},\label{eq:sos:alpha}\\
	\beta_{ijk}(-\omega_\sigma; \omega_1, \omega_2) &= \hbar^{-2} \sum_\mathcal{P} \sum_{a_1,a_2} \frac{(\zeta\eta\kappa)_{a_1 a_2}}{(\omega_{a_1} - \omega_\sigma)(\omega_{a_2} - \omega_\sigma + \omega_1)},\label{eq:sos:beta}\\
	\gamma_{ijkl}(-\omega_\sigma; \omega_1, \omega_2, \omega_3) &= \hbar^{-3} \sum_\mathcal{P} \sum_{a_1, a_2, a_3} \frac{(\zeta\eta\kappa\lambda)_{a_1 a_2 a_3}}{(\omega_{a_1} - \omega_\sigma)(\omega_{a_2} - \omega_\sigma + \omega_1)(\omega_{a_3} - \omega_\sigma + \omega_1 + \omega_2)},\label{eq:sos:gamma}
\end{align}
representing the polarizability $\alpha = \chi^{(1)}$, first hyperpolarizability $\beta = \chi^{(2)}$, and second hyperpolarizability $\gamma = \chi^{(3)}$. Here, the numerator notation of Bishop \cite{bishopExplicitNondivergentFormulas1994}, $(\zeta\eta\kappa\lambda)_{a_1 a_2 a_3} = \mu_{0 a_1}^\zeta \mu_{a_1 a_2}^\eta \mu_{a_2 a_3}^\kappa \mu_{a_3 0}^\lambda$, is employed.

Each of Eqs.~\eqref{eq:sos:alpha}-\eqref{eq:sos:gamma} encounters divergences (or singularities) when a denominator vanishes. This phenomenon is termed \textbf{secular divergence} if caused by any state $\ket{a_i} = \ket{0}$ (and thus $\omega_{a_i} = 0$) or, if any optical frequency (or a combination thereof) matches $\omega_{a_i} \neq 0$, is generally termed \textbf{resonance} \cite{bishopExplicitNondivergentFormulas1994}. While resonances are intrinsic to perturbation theory and often mitigated by introducing damping factors (though methods remain debated \cite{campoPracticalModelFirst2012a}), secular divergences are mathematical artifacts and can be avoided. 

Following Bishops, substituting the dipole operator in Eq.~\eqref{eq:1:sos} with a fluctuation dipole operator, $\bar{\mu}^\zeta_{a_1 a_2} = \mu^\zeta_{a_1 a_2} - \delta_{a_1 a_2}\, \mu_{00}^\zeta$, results in $(\bar{\zeta})_g = 0$, allowing the ground state to be excluded from the summations in Eqs.~\eqref{eq:sos:alpha} and \eqref{eq:sos:beta}. Consequently, we obtain:
\begin{align}
	\alpha_{ij}(-\omega; \omega) &= \hbar^{-1} \sum_\mathcal{P} \sum_{a_1'} \frac{(\zeta\eta)_{a_1}}{\omega_{a_1} - \omega},\label{eq:fluct:alpha}\\
	\beta_{ijk}(-\omega_\sigma; \omega_1, \omega_2) &= \hbar^{-2} \sum_\mathcal{P} \sum_{a_1', a_2'} \frac{(\zeta\bar{\eta}\kappa)_{a_1 a_2}}{(\omega_{a_1} - \omega_\sigma)(\omega_{a_2} - \omega_\sigma + \omega_1)},\label{eq:fluct:beta}
\end{align}
where the prime indicates that the sums over $a_1$ (and $a_2$) now exclude $\ket{0}$. This adjustment removes secular divergence, but also permits the first and last transition dipoles in each term to omit the ``bar'' as well.

Applying this procedure to Eq.~\eqref{eq:sos:gamma} introduces an error in cases where terms with $\ket{a_2} = \ket{0}$ are omitted. The correct expression for the second hyperpolarizability, $\gamma$, is therefore the sum of two components, $\gamma = \gamma^{(+)} + \gamma^{(-)}$, where:
\begin{align}
	\gamma_{ijkl}^{(+)}(-\omega_\sigma; \omega_1, \omega_2, \omega_3) &= \hbar^{-3} \sum_\mathcal{P} \sum_{a_1', a_2', a_3'} \frac{(\zeta \bar{\eta} \bar{\kappa} \lambda)_{a_1 a_2 a_3}}{(\omega_{a_1} - \omega_\sigma)(\omega_{a_2} - \omega_\sigma + \omega_1)(\omega_{a_3} - \omega_\sigma + \omega_1 + \omega_2)}, \nonumber\\
	\gamma_{ijkl}^{(-)}(-\omega_\sigma; \omega_1, \omega_2, \omega_3) &= \hbar^{-3} \sum_\mathcal{P} \sum_{a_1', a_3'} \frac{(\zeta \eta)_{a_1} (\kappa \lambda)_{a_3}}{(\omega_{a_1} - \omega_\sigma)(-\omega_\sigma + \omega_1)(\omega_{a_3} - \omega_\sigma + \omega_1 + \omega_2)},
\end{align}
where $\gamma^{(+)}$ corresponds to the expression when summing over all non-ground states, while $\gamma^{(-)}$ is a correction term, obtained by setting $\ket{a_2} = \ket{0}$ in the expression of $\gamma^{(+)}$. However, these (so-called) secular terms, grouped in $\gamma^{(-)}$, lead to divergence if the conditions $-\omega_\sigma + \omega_1 = \omega_2 + \omega_3 = 0$ is satisfied, even though the ground state is excluded from the summation.

Before addressing this divergence in detail, note that a generalization of this procedure yields \begin{equation}
	\chi^{(n)}_{\zeta \eta \ldots \nu}(-\omega_\sigma; \omega_1, \ldots)  = \chi^{(n,+)}_{\zeta \eta \ldots \nu}(-\omega_\sigma; \omega_1, \ldots)  + \chi^{(n,-)}_{\zeta \eta \ldots \nu}(-\omega_\sigma; \omega_1, \ldots) ,\label{eq:fluct}
\end{equation} where $\chi^{(n,+)}$ represents the non-secular contributions, given by:
\begin{align}
	\chi^{(n,+)}_{\zeta \eta \ldots \nu}(-\omega_\sigma; \omega_1, \ldots) = \hbar^{-n} \sum_\mathcal{P} \sum_{a_1', a_2', \ldots} \frac{(\zeta \bar{\eta} \bar{\kappa} \ldots \nu)_{a_1 a_2 \ldots a_n}}{\prod_{0 < i \leq n} \omega_{a_i}'} ,\label{eq:fluct:nonsecular}
\end{align}
which follows directly from Eq.~\eqref{eq:1:sos}, with the summation now excluding the ground state. Here, the notation $\omega_{a_i}' = \omega_{a_i} - \omega_\sigma + \sum_{0 < j < i} \omega_j$ is introduced, which will be useful next.
The secular contributions, $\chi^{(n,-)}$, are given by:
\begin{align}
	\chi^{(n,-)} = \sum_{1 < i < n} \left[ \left. \chi^{(n,+)} \right|_{\ket{a_i} = \ket{0}} + \sum_{i+1 < j < n} \left( \left. \chi^{(n,+)} \right|_{\ket{a_i} = \ket{a_j} = \ket{0}} + \ldots \right) \right],\label{eq:fluct:secular}
\end{align}
where Cartesian indices and laser frequencies have been omitted for clarity. The notation $\ket{a_i} = \ket{a_j} = \ket{0}$ specifies that both states $\ket{a_i}$ and $\ket{a_j}$ are evaluated as the ground state in Eq.~\eqref{eq:fluct:nonsecular}. The number of secular terms increases with $n$, as higher-order interactions introduce additional configurations in which intermediate states are the ground state. 

After theses formula have been implemented, comparisons between Eq.~\eqref{eq:1:sos} and Eq.~\eqref{eq:fluct} have been successfully conducted to get tensors, $\chi^{(n)}$ corresponding to $n$\textsuperscript{th}-harmonic generation process (where all $\omega_i = \omega$), up to $n = 6$ (which includes a secular term where $\ket{a_2} = \ket{a_4} = \ket{0}$), within both two- and three-state models.

\subsection{Non-divergent secular terms}

According to Bishop \cite{bishopExplicitNondivergentFormulas1994}, to avoid divergence in secular terms, one can use the fact that they are invariant to time-reversal, so that:\begin{equation*}
	\chi^{(n,-)}_{\zeta \eta \ldots \nu}(-\omega_\sigma; \omega_1, \omega_2, \ldots)  = \chi^{(n,-)}_{\zeta \eta \ldots \nu}(+\omega_\sigma; -\omega_1, -\omega_2,\ldots),
\end{equation*}
as it is the case for any NLO tensor element. The idea is thus to write $\chi^{(n,-)}$ as an average of itself and its time-reversal version. Doing so for $\left. \chi^{(n,+)} \right|_{\ket{a_j} = \ket{0}}$ (thus $\omega_{a_j} = 0$) and focusing on the denominator (the numerator is not affected, as the dipole operator is also invariant to time reversal) gives:\begin{align*}
 	\frac{1}{2x}\left[
	 \frac{1}{\prod_{0 < i\neq j \leq n} (\omega_{a_i}'+x) } - \frac{1}{{\prod_{0 < i\neq j \leq n} \omega_{a_i}'}}
	\right] = -\frac{1}{2x}\left[
	\frac{\prod_{0 < i\neq j \leq n} (\omega_{a_i}'+x) - \prod_{0 < i\neq j \leq n} \omega_{a_i}}{\prod_{0 < i\neq j \leq n} (\omega_{a_i}'+x)\,\omega_{a_i}'}
	\right],
\end{align*}
after defining $\omega_{a_j}'=-x$ for bookkeeping, and thanks to the permutation of a few indices to obtain $\omega_{a_i}'-x$ in the denominator of the time-reversal version. 


\appendix
\section{Proofs}

\paragraph{Simplification leading to Eq.~x.} Given a set of reals $k_i$ so that $\{k_i|0<i\leq n\}$, \begin{equation}
	\prod_{0<i\leq n} (k_i+x) =  \prod_{0<i\leq n}( k_i)+ x\,\left[\sum_{0< i\leq n} \left(\prod_{0<j< i}(k_j+x)\prod_{i<j\leq n}  (k_j)\right)\right].\label{eq:p1}
\end{equation}
Proof: in preamble, let us remind that for any $0<i\leq n$, one has:\begin{align}
	\prod_{i<j\leq n} (k_j+x)&= (k_{i+1}+x)\,\prod_{i+1<j\leq n} (k_j+x)\nonumber\\
	&= k_{i+1}\,\left[\prod_{i+1<j\leq n} (k_j+x) \right] + x\,\left[\prod_{i+1<j\leq n} (k_j+x) \right].\label{eq:p1:ind}
\end{align}
Now, let's prove Eq.~\eqref{eq:p1}. The $n=0$ and $n=1$ cases are trivial. For $n=2$, using Eq.~\eqref{eq:p1:ind} two times then rearranging:\begin{align*}
	f(x) &= (k_1+x) \,(k_2+x)\\
	&= k_2\,(k_1+x)+ x\,(k_1+x)\\
	&= k_1k_2+x\,k_2+ x\,(k_1+x) = k_1k_2+x\,[(k_1+x)+ k_2],
\end{align*}
which fulfill Eq.~\eqref{eq:p1}.
For $n=3$, using Eq.~\eqref{eq:p1:ind} three times then rearranging:\begin{align*}
	f(x) &= (k_1+x) \,(k_2+x)\,(k_3+x)\\
	&= k_3\,(k_1+x) \,(k_2+x)+x\,(k_1+x)\,(k_2+x)\\
	&= k_2\,k_3\,(k_1+x)+x\,k_3\,(k_1+x)+x\,(k_1+x)\,(k_2+x)\\
	&= k_1k_2k_3+x\,k_2\,k_3+x\,k_3\,(k_1+x)+x\,(k_1+x)\,(k_2+x)\\
	&= k_1k_2k_3+x\,[(k_1+x)\,(k_2+x)+(k_1+x)\,k_3+k_2\,k_3]
	\end{align*}
which also fulfill Eq.~\eqref{eq:p1}. Finally, given the case $n=N$, let's prove for $n=N+1$:
\begin{align*}
	\prod_{0<i\leq N+1} (k_i+x) &= (k_{N+1}+x)\, \prod_{0<i\leq N} (k_i+x) \\
	&= k_{N+1}\, \prod_{0<i\leq N} (k_i+x)  + x\,\prod_{0<i\leq N} (k_i+x)\\
	&= k_{N+1}\,\left\{   \prod_{0<i\leq N}( k_i)+ x\,\left[\sum_{0< i\leq N} \left(\prod_{0<j< i}(k_j+x)\prod_{i<j\leq N}  (k_j)\right)\right]\right\} + x\,\prod_{0<i\leq N} (k_i+x)\\
	&= \prod_{0<i\leq N+1}( k_i)+ x\,k_{N+1}\,\left[\sum_{0< i\leq N} \left(\prod_{0<j< i}(k_j+x)\prod_{i<j\leq N}  (k_j)\right)\right] + x\,\prod_{0<i\leq N} (k_i+x)\\
	&= \prod_{0<i\leq N+1}( k_i)+ x\,\left[\sum_{0< i\leq N} \left(\prod_{0<j< i}(k_j+x)\prod_{i<j\leq N+1}  (k_j)\right)\right] + x\,\prod_{0<i< N+1} (k_i+x)\\
	&= \prod_{0<i\leq N+1}( k_i)+ x\,\left[\sum_{0< i\leq N+1} \left(\prod_{0<j< i}(k_j+x)\prod_{i<j\leq N+1}  (k_j)\right)\right].
\end{align*}
The three last lines have been obtained by carefully rewriting the boundaries of the sums and products.
Thus, by induction, Eq.~\eqref{eq:p1} is valid for any $n\geq0$.
	
\bibliographystyle{unsrt}
\bibliography{biblio}
	
\end{document}