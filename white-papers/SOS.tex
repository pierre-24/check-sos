\documentclass[12pt,a4paper]{article}
\usepackage[T1]{fontenc}
\usepackage[left=2cm, right=2cm, top=2cm, bottom=2cm]{geometry}
\usepackage{graphicx}
\usepackage{mathtools}
\usepackage{amssymb}
\usepackage{amsthm}
\usepackage{hyperref}
\usepackage{braket}
\usepackage{setspace}

\title{On sum over states (SOS)}
\author{Pierre Beaujean}
\begin{document}
\maketitle

\allowdisplaybreaks
\onehalfspacing

\section{Introduction}

On the ground of perturbation theory, 
the SOS expression of Orr and Ward \cite{orrPerturbationTheoryNonlinear1971} (see also Bishop \cite{bishopExplicitNondivergentFormulas1994}) states that, far from resonance, any component of any nonlinear optical tensor $X^{(n)}(-\omega_\sigma;\omega_1,\ldots)$ (of order $n$) is given by:\begin{align}
&X^{(n)}_{\zeta\eta\ldots\nu}(-\omega_\sigma;\omega_1,\ldots) = \hbar^{-n}\sum_{\mathcal{P}_F}\sum_{a_1,a_2\ldots\,a_{n}} \frac{\mu^\zeta_{0a_1}{\mu}^\eta_{a_1a_2}\ldots \mu^\nu_{a_{n\,0}}}{\prod_{0<i\leq n} (\omega_{a_i}-\omega_\sigma+\sum_{0<j<i} \omega_j)},\label{eq:1:sos}
\end{align}
where $\zeta,\eta,\ldots$ are the Cartesian coordinates $x, y, z$ (in the molecular frame), $\omega_1, \omega_2\ldots$, the (optical) input frequencies of the laser for the NLO process (with $\omega_\sigma = \sum_{0<i<n} \omega_i$), $\ket{a_1}, \ket{a_2}, \ldots$, the states of the system  \textbf{including the ground state} (with $\hbar\omega_{a_i}$ the excitation energy from ground state, noted  $\ket{0}$, to $\ket{a_i}$), $\mu^\zeta_{a_ia_j} = \braket{a_i|\hat \zeta|a_j}$ the transition dipole moment from state $a_i$ to $a_j$ (it corresponds to the dipole moment of electronic state $a_i$ when $i=j$), and $\sum_{\mathcal{P}_F}$ the sum of all permutations over each pair $(\zeta, \omega_\sigma),(\eta,\omega_1),\ldots$. 

Given the form of Eq.~\eqref{eq:1:sos}, it is relatively easy to write a (Python) code that compute any $X^{(n)}$. However, doing so requires care, since this expression blows up in many cases. The goal of this document is to find alternative formulas, while retaining generality.

\section{Theory}

Examining the expressions for $n\in[1,3]$ more closely, one has:
\begin{align}
	\alpha_{\zeta\eta}(-\omega;\omega) &= \hbar^{-1} \sum_{\mathcal{P}_F} \sum_{a_1} \frac{(\zeta\eta)_{a_1}}{\omega_{a_1} - \omega},\label{eq:sos:alpha}\\
	\beta_{\zeta\eta\kappa}(-\omega_\sigma; \omega_1, \omega_2) &= \hbar^{-2} \sum_{\mathcal{P}_F} \sum_{a_1,a_2} \frac{(\zeta\eta\kappa)_{a_1 a_2}}{(\omega_{a_1} - \omega_\sigma)(\omega_{a_2} - \omega_\sigma + \omega_1)},\label{eq:sos:beta}\\
	\gamma_{\zeta\eta\kappa\lambda}(-\omega_\sigma; \omega_1, \omega_2, \omega_3) &= \hbar^{-3} \sum_{\mathcal{P}_F} \sum_{a_1, a_2, a_3} \frac{(\zeta\eta\kappa\lambda)_{a_1 a_2 a_3}}{(\omega_{a_1} - \omega_\sigma)(\omega_{a_2} - \omega_\sigma + \omega_1)(\omega_{a_3} - \omega_\sigma + \omega_1 + \omega_2)},\label{eq:sos:gamma}
\end{align}
representing the polarizability $\alpha = X^{(1)}$, first hyperpolarizability $\beta = X^{(2)}$, and second hyperpolarizability $\gamma = X^{(3)}$. Here, the numerator notation of Bishop \cite{bishopExplicitNondivergentFormulas1994}, $(\zeta\eta\kappa\lambda)_{a_1 a_2 a_3} = \mu_{0 a_1}^\zeta \mu_{a_1 a_2}^\eta \mu_{a_2 a_3}^\kappa \mu_{a_3 0}^\lambda$, is employed.

Each of Eqs.~\eqref{eq:sos:alpha}-\eqref{eq:sos:gamma} encounters divergences (or singularities) when a denominator vanishes. This phenomenon is termed \textbf{secular divergence} if caused by any state $\ket{a_i} = \ket{0}$ (and thus $\omega_{a_i} = 0$) or, if any optical frequency (or a combination thereof) matches $\omega_{a_i} \neq 0$, is generally termed \textbf{resonance} \cite{bishopExplicitNondivergentFormulas1994}. While resonances are intrinsic to perturbation theory and often mitigated by introducing damping factors (though methods remain debated \cite{campoPracticalModelFirst2012a}), secular divergences are mathematical artifacts and can be avoided. 

\subsection{Avoiding secular divergence: using fluctuation dipole}

Following Orr and Ward \cite{orrPerturbationTheoryNonlinear1971} and latter Bishop \cite{bishopExplicitNondivergentFormulas1994}, substituting the dipole operator in Eq.~\eqref{eq:1:sos} with a fluctuation dipole operator, $\bar{\mu}^\zeta_{a_1 a_2} = \mu^\zeta_{a_1 a_2} - \delta_{a_1 a_2}\, \mu_{00}^\zeta$, results in $(\bar{\zeta})_g = 0$, allowing the ground state to be excluded from the summations in Eqs.~\eqref{eq:sos:alpha} and \eqref{eq:sos:beta}. Consequently, we obtain:
\begin{align*}
	\alpha_{\zeta\eta}(-\omega; \omega) &= \hbar^{-1} \sum_{\mathcal{P}_F} \sum_{a_1'} \frac{(\zeta\eta)_{a_1}}{\omega_{a_1} - \omega},\\
	\beta_{\zeta\eta\kappa}(-\omega_\sigma; \omega_1, \omega_2) &= \hbar^{-2} \sum_{\mathcal{P}_F} \sum_{a_1', a_2'} \frac{(\zeta\bar{\eta}\kappa)_{a_1 a_2}}{(\omega_{a_1} - \omega_\sigma)(\omega_{a_2} - \omega_2)},
\end{align*}
where the prime indicates that the sums over $a_1$ (and $a_2$) now exclude $\ket{0}$. This adjustment removes secular divergence, but also permits the first and last transition dipoles in each term to omit the ``bar'' as well.

Applying this procedure to Eq.~\eqref{eq:sos:gamma} introduces an error in cases where terms with $\ket{a_2} = \ket{0}$ are omitted. The correct expression for the second hyperpolarizability, $\gamma$, is therefore the sum of two components, $\gamma = \gamma^{(+)} + \gamma^{(-)}$, where:
\begin{align}
	\gamma_{\zeta\eta\kappa\lambda}^{(+)}(-\omega_\sigma; \omega_1, \omega_2, \omega_3) &= \hbar^{-3} \sum_{\mathcal{P}_F} \sum_{a_1', a_2', a_3'} \frac{(\zeta \bar{\eta} \bar{\kappa} \lambda)_{a_1 a_2 a_3}}{(\omega_{a_1} - \omega_\sigma)(\omega_{a_2} - \omega_\sigma + \omega_1)(\omega_{a_3} - \omega_\sigma + \omega_1 + \omega_2)}, \nonumber\\
	\gamma_{\zeta\eta\kappa\lambda}^{(-)}(-\omega_\sigma; \omega_1, \omega_2, \omega_3) &= \hbar^{-3} \sum_{\mathcal{P}_F} \sum_{a_1', a_3'} \frac{(\zeta \eta)_{a_1} (\kappa \lambda)_{a_3}}{(\omega_{a_1} - \omega_\sigma)(-\omega_\sigma + \omega_1)(\omega_{a_3} - \omega_\sigma + \omega_1 + \omega_2)},
	\label{eq:fluct:gamma}
\end{align}
where $\gamma^{(+)}$ corresponds to the expression when summing over all non-ground states, while $\gamma^{(-)}$ is a correction term, obtained by setting $\ket{a_2} = \ket{0}$ in the expression of $\gamma^{(+)}$. However, these (so-called) secular terms, grouped in $\gamma^{(-)}$, lead to divergence if the conditions $-\omega_\sigma + \omega_1 = \omega_2 + \omega_3 = 0$ is satisfied, even though the ground state is excluded from the summation.

Before addressing this divergence in detail, note that a generalization of this procedure yields \begin{equation}
	X^{(n)}_{\zeta \eta \ldots \nu}(-\omega_\sigma; \omega_1, \ldots)  = X^{(n,+)}_{\zeta \eta \ldots \nu}(-\omega_\sigma; \omega_1, \ldots)  + X^{(n,-)}_{\zeta \eta \ldots \nu}(-\omega_\sigma; \omega_1, \ldots) ,\label{eq:fluct}
\end{equation} where $X^{(n,+)}$ represents the non-secular contributions, given by:
\begin{align}
	X^{(n,+)}_{\zeta \eta \ldots \nu}(-\omega_\sigma; \omega_1, \ldots) = \hbar^{-n} \sum_{\mathcal{P}_F} \sum_{a_1', a_2', \ldots} \frac{(\zeta \bar{\eta} \bar{\kappa} \ldots \nu)_{a_1 a_2 \ldots a_n}}{\prod_{0 < i \leq n} \omega_{a_i}'} ,\label{eq:fluct:nonsecular}
\end{align}
which follows directly from Eq.~\eqref{eq:1:sos}, with the summation now excluding the ground state. Here, the notation $\omega_{a_i}' = \omega_{a_i} - \omega_\sigma + \sum_{0 < j < i} \omega_j$ is introduced, which will be useful next.
The secular contributions, $X^{(n,-)}$, are given by:
\begin{align}
	X^{(n,-)} = \sum_{1 < i < n} \left[ \left. X^{(n,+)} \right|_{\ket{a_i} = \ket{0}} + \sum_{i+1 < j < n} \left( \left. X^{(n,+)} \right|_{\ket{a_i} = \ket{a_j} = \ket{0}} + \ldots \right) \right],\label{eq:fluct:secular}
\end{align}
where Cartesian indices and laser frequencies have been omitted for clarity. The notation $\ket{a_i} = \ket{a_j} = \ket{0}$ specifies that both states $\ket{a_i}$ and $\ket{a_j}$ are evaluated as the ground state in Eq.~\eqref{eq:fluct:nonsecular}. The number of secular terms increases with $n$, as higher-order interactions introduce additional configurations in which intermediate states are the ground state. 

\subsection{Curing the remaining secular terms}

To avoid divergence in secular terms, Bishop \cite{bishopExplicitNondivergentFormulas1994} suggests using the fact that these terms are invariant under time-reversal, so that:
\begin{equation*}
	X^{(n,-)}_{\zeta \eta \ldots \nu}(-\omega_\sigma; \omega_1, \omega_2, \ldots) = X^{(n,-)}_{\zeta \eta \ldots \nu}(+\omega_\sigma; -\omega_1, -\omega_2, \ldots),
\end{equation*}
a property characteristic of any nonlinear optical (NLO) tensor element. The approach, then, is to rewrite $X^{(n,-)}$ as the average of itself and its time-reversed counterpart. Applying this procedure to $\left. X^{(n,+)} \right|_{\ket{a_j} = \ket{0}}$ (where $\omega_{a_j} = 0$) and focusing on the denominator (the numerator remains unaffected due to the time-reversal invariance of the dipole operator) yields:
\begin{align*}
	\frac{1}{2x} \left[ \frac{1}{\prod_{0 < i \neq j \leq n} (\omega_{a_i}' + x)} - \frac{1}{\prod_{0 < i \neq j \leq n} \omega_{a_i}'} \right] 
	&= -\frac{1}{2x} \left[ \frac{\prod_{0 < i \neq j \leq n} (\omega_{a_i}' + x) - \prod_{0 < i \neq j \leq n} \omega_{a_i}'}{\prod_{0 < i \neq j \leq n} (\omega_{a_i}' + x) \, \omega_{a_i}'} \right],
\end{align*}
after setting $\omega_{a_j}' = -x$ for convenience and using a permutation of indices to obtain $\omega_{a_i}' - x$ in the denominator of the time-reversed term. Applying Theorem~\ref{th:1} to the expression above gives:
\begin{align*}
	-\frac{1}{2} \left[ \frac{\sum_{0 < i \neq j \leq n} \left( \prod_{0 < l \neq j < i} (\omega'_{a_l} + x) \prod_{i < l \neq j \leq n} (\omega'_{a_l}) \right)}{\prod_{0 < l \neq j \leq n} (\omega_{a_l}' + x) \, \omega_{a_l}'} \right] 
	&= -\frac{1}{2} \sum_{0 < i \neq j \leq n} \frac{1}{\prod_{0 < l \neq j \leq i} (\omega'_{a_l}) \prod_{i \leq l \neq j \leq n} (\omega'_{a_l} + x)}.
\end{align*}
Thus, we obtain:
\begin{equation}
	\left. X^{(n,+)} \right|_{\ket{a_j} = \ket{0}} = -\frac{1}{2\hbar^n} \sum_{\mathcal{P}_F} \sum_{a_1', a_2', \ldots} \sum_{0 < i \neq j \leq n} \frac{(\zeta \bar{\eta} \ldots \kappa)_{a_1 a_2 \ldots a_{j-1}} (\xi \bar{\tau} \ldots \nu)_{a_{j+1} \ldots a_n}}{\prod_{0 < l \neq j \leq i} (\omega'_{a_l}) \prod_{i \leq l \neq j \leq n} (\omega'_{a_l} + x)}, \label{eq:sec:nondiv}
\end{equation}
where $x = \omega_\sigma - \sum_{0 < l < j} \omega_l$. By further manipulating this equation, one can derive Eq.~\eqref{eq:fluct:gamma}. Additional explicit formulas are provided in Ref.~\cite{bishopExplicitNondivergentFormulas1994}. However, Eq.~\eqref{eq:sec:nondiv} is general enough to be implemented in a Python code.

With Eq.~\eqref{eq:sec:nondiv}  alone, one is limited to $n<5$, since $n=5$ and above requires to include configuration where two intermediate states are ground. While it is most definitely possible to derive an expression for such case, it is out of the scope for the moment.

\subsection{Resonances}

\begin{equation}\frac{2 i \Gamma m_{00} m_{01}^{2}}{\omega \left(\Gamma^{2} - 2 i \Gamma \omega - \omega^{2} + \omega_{ge}^{2}\right)}
\end{equation}



To avoid resonance effects, the simplest approach is to augment each excitation energy with an imaginary component, $\Omega_{a_n} = \omega_{a_n} - i\Gamma$, where $\Gamma$ represents the inverse lifetime of the excited state.\footnote{In fact, perturbation theory starts by assuming complex excitation energies and proceed from Eq.~\eqref{eq:resonant:fluct:nonsecular} to Eq.~\eqref{eq:fluct:nonsecular}, not the other way around, but... Well.} It is, however, not possible to starts from Eq.~\eqref{eq:1:sos}, since this would means that the ground state has a lifetime as well (this model could, however, be extended by assigning a distinct lifetime for each state, $\Gamma_n$). 
Thus, using fluctation dipoles, the expression corresponding to Eq.~\eqref{eq:fluct:nonsecular} is:\begin{equation}
X^{(n,+)}_{\zeta \eta \kappa \ldots \nu}(-\omega_\sigma; \omega_1, \ldots) = \hbar^{-n} \sum_{\mathcal{P}_I}\sum_{\mathcal{P}_C} \sum_{a_1', a_2', \ldots} \frac{(\zeta \bar{\eta} \bar{\kappa} \ldots \nu)_{a_1 a_2 \ldots a_n}}{\prod_{0 < i \leq \mathcal C} (\Omega_{a_i}')^\star \prod_{\mathcal C < i \leq n} \Omega_{a_i}'},\label{eq:resonant:fluct:nonsecular}
\end{equation}
where $\Omega_{a_i}' = \omega_{a_i} - i\Gamma - \omega_\sigma + \sum_{0 < j < i} \omega_j$.
In this expression, $\sum_{\mathcal{P}_I}$ represents the sum over intrinsic permutations, which corresponds to $\sum_{\mathcal{P}_F}$ but excludes the first pair, summing instead over permutations of $(\eta, \omega_1), (\kappa, \omega_2), \ldots$ and $\sum_{\mathcal{P}_C}$ denotes the sum over \textbf{cyclic permutations} of the pairs $(\zeta, \omega_\sigma), (\eta, \omega_1), \ldots$, with the cycle count labeled by $\mathcal{C}$.
For example, when $n = 2$, the cyclic permutations are:
\begin{equation*}
\underbrace{(\zeta,\omega_\sigma),(\eta,\omega_1),(\kappa,\omega_2)}_{\mathcal{C} = 0} \rightarrow \underbrace{(\eta,\omega_1),(\kappa,\omega_2),(\zeta,\omega_\sigma)}_{\mathcal{C} = 1} \rightarrow \underbrace{(\kappa,\omega_2),(\zeta,\omega_\sigma),(\eta,\omega_1)}_{\mathcal{C} = 2}.
\end{equation*}
When $\Gamma = 0$, the original equation, Eq.~\eqref{eq:fluct:nonsecular}, is recovered because the relation $\sum_{\mathcal{P}_F} = \sum_{\mathcal{P}_I} \sum_{\mathcal{P}_C}$ holds.\footnote{Any permutation can be decomposed into a product of (disjoint) cycles} 

In practice, for $n=1$, Eq.~\eqref{eq:resonant:fluct:nonsecular} gives explicitly:\begin{equation*}
	\alpha_{\zeta\eta}(-\omega; \omega) = \hbar^{-1} \sum_{a_1'} \left(\frac{(\zeta\eta)_{a_1}}{\Omega_{a_1} - \omega}+ \frac{(\eta\zeta)_{a_1}}{\Omega_{a_1}^\star + \omega}\right),
	\end{equation*}
	and, for $n=2$:
	\begin{align*}
	\beta_{\zeta\eta\kappa}(-\omega_\sigma; \omega_1, \omega_2) = \hbar^{-2} \sum_{\mathcal{P}_I} \sum_{a_1', a_2'} &\left(\frac{(\zeta\bar{\eta}\kappa)_{a_1 a_2}}{(\Omega_{a_1} - \omega_\sigma)(\Omega_{a_2} - \omega_2)}+ \frac{(\eta\bar\kappa\zeta)_{a_1 a_2}}{(\Omega_{a_1}^\star +\omega_2)(\Omega_{a_2} -\omega_1)}\right.\\
	&\hspace{2em}+\left.\frac{(\kappa\bar\zeta\eta)_{a_1 a_2}}{(\Omega_{a_1}^\star + \omega_1)(\Omega_{a_2}^\star + \omega_\sigma)} \right).
\end{align*}

For $n>2$, one again needs to include the extra secular terms, and since  Theorem~\ref{th:1}  remain valid, one obtains the following non-divergent version:\begin{align}
	&\left. X^{(n,+)} \right|_{\ket{a_j} = \ket{0}} = -\frac{1}{2\hbar^n} \sum_{\mathcal{P}_I} \sum_{\mathcal{P}_C}\sum_{a_1', a_2', \ldots} \sum_{0 < i \neq j \leq n} \nonumber\\
	&\hspace{2em}\frac{(\zeta \bar{\eta} \ldots \kappa)_{a_1 a_2 \ldots a_{j-1}} (\xi \bar{\tau} \ldots \nu)_{a_{j+1} \ldots a_n}}{\prod_{ \mathcal C \leq i \land 0 < l \neq j \leq \mathcal C } (\Omega'_{a_l})^\star \prod_{ \mathcal C < l \neq j \leq i} (\Omega'_{a_l}) \prod_{i \leq l \neq j \leq  \mathcal C} (\Omega'_{a_l} + x)^\star \prod_{ \mathcal C > i \land \mathcal C \leq l \neq j \leq  n} (\Omega'_{a_l} + x)}, \label{eq:resonant:sec:nondiv}
\end{align}
which, while a bit convoluted,\footnote{And not exactly readable.} provide the correct expression for such extra terms. $x$ retains the same meaning as in Eq.~\eqref{eq:sec:nondiv}. However, (???).

\section{Results}

For the non-resonant version, after theses formula have been implemented, comparisons between Eq.~\eqref{eq:1:sos} and Eq.~\eqref{eq:fluct} have been successfully conducted to get tensors, $X^{(n)}$ corresponding to $n$\textsuperscript{th}-harmonic generation process (where all $\omega_i = \omega$), within both two- and three-state models.
It was performed up to  $n = 5$ with divergent secular terms (which includes a secular term where $\ket{a_2} = \ket{a_4} = \ket{0}$), and up to $n=4$ with non-divergent secular terms [using Eq.~\eqref{eq:sec:nondiv}]. Then, different tensors corresponding to NLO processes which include a static field where tested with $n=3$ (static, dc-kerr, DFWM, and EFISHG) using Eq.~\eqref{eq:sec:nondiv} and a three-state model. No sign of divergence where found.

The resonant version was also implemented [using Eqs.~\eqref{eq:resonant:fluct:nonsecular} and \eqref{eq:resonant:sec:nondiv} in Eq.~\eqref{eq:fluct}], and compared to the work of Berkovic \textit{et al.} for SHG \cite{berkovicMeasurementAnalysisMolecular2000}. It was also compared to non-resonant version when setting $\Gamma = 0$.



\bibliographystyle{unsrt}
\bibliography{biblio}

\clearpage
\appendix
\section{A few proofs}
\setcounter{equation}{0} 
\renewcommand{\theequation}{A\arabic{equation}}

\newtheorem{theorem}{Theorem}[section]
\newtheorem{lemma}[theorem]{Lemma}

\begin{lemma}\label{lem:1}
	Given a set of reals $k_i$ so that $\{k_i|0<i\leq n\}$,  for any $0<i\leq n$, one has:\begin{align}
		\prod_{0<j\leq i} (k_j+x)= k_{i}\,\left[\prod_{0<j<i} (k_j+x) \right] + x\,\left[\prod_{0<j<i} (k_j+x) \right].\label{eq:p1:ind}
	\end{align}
\end{lemma}
\begin{proof}
	Trivial.
\end{proof}

\begin{theorem}\label{th:1}
	Given a set of reals $k_i$ so that $\{k_i|0<i\leq n\}$, \begin{equation}
		\prod_{0<i\leq n} (k_i+x) =  \prod_{0<i\leq n}( k_i)+ x\,\left[\sum_{0< i\leq n} \left(\prod_{0<j< i}(k_j+x)\prod_{i<j\leq n}  (k_j)\right)\right].\label{eq:p1}
	\end{equation}
\end{theorem}
\begin{proof}
	The $n=0$ and $n=1$ cases are trivial. For $n=2$, using Lemma \ref{lem:1} two times then rearranging:\begin{align*}
		f(x) &= (k_1+x) \,(k_2+x)\\
		&= k_2\,(k_1+x)+ x\,(k_1+x)\\
		&= k_1k_2+x\,k_2+ x\,(k_1+x) = k_1k_2+x\,[(k_1+x)+ k_2],
	\end{align*}
	fulfills theorem \ref{th:1}.
	For $n=3$, using Lemma \ref{lem:1}, then the result for $n=2$ gives:\begin{align*}
		f(x) &= (k_1+x) \,(k_2+x)\,(k_3+x) \\
		&= k_3\,(k_1+x) \,(k_2+x)+x\,(k_1+x)(k_2+x) \\
		&= k_3\,\{k_1k_2+x\,[(k_1+x)+ k_2]\}+x\,(k_1+x)(k_2+x) \\
		&= k_1k_2k_3+x\,[(k_1+x)(k_2+x)+ (k_1+x)\,k_3+ k_2k_3]
	\end{align*}
	which also fulfill  theorem \ref{th:1}. Finally, given the case $n=N$, let's prove for $n=N+1$:
	\begin{align*}
		\prod_{0<i\leq N+1} (k_i+x) &= k_{N+1}\, \prod_{0<i\leq N} (k_i+x)  + x\,\prod_{0<i\leq N} (k_i+x)\\
		&= k_{N+1}\,\left\{   \prod_{0<i\leq N}( k_i)+ x\,\left[\sum_{0< i\leq N} \left(\prod_{0<j< i}(k_j+x)\prod_{i<j\leq N}  (k_j)\right)\right]\right\} + x\,\prod_{0<i\leq N} (k_i+x)\\
		&= \prod_{0<i\leq N+1}( k_i)+ x\,k_{N+1}\,\left[\sum_{0< i\leq N} \left(\prod_{0<j< i}(k_j+x)\prod_{i<j\leq N}  (k_j)\right)\right] + x\,\prod_{0<i\leq N} (k_i+x)\\
		&= \prod_{0<i\leq N+1}( k_i)+ x\,\left[\sum_{0< i\leq N} \left(\prod_{0<j< i}(k_j+x)\prod_{i<j\leq N+1}  (k_j)\right)\right] + x\,\prod_{0<i< N+1} (k_i+x)\\
		&= \prod_{0<i\leq N+1}( k_i)+ x\,\left[\sum_{0< i\leq N+1} \left(\prod_{0<j< i}(k_j+x)\prod_{i<j\leq N+1}  (k_j)\right)\right].
	\end{align*}
	The first line is the application of Lemma \ref{lem:1}, and the second uses Eq.~\eqref{eq:p1} for $n=N$. The three last lines have been obtained by carefully rewriting the boundaries of the sums and products, following the development for $n=3$.
	Thus, by induction,  theorem \ref{th:1} is valid for any $n\geq0$.
\end{proof}
	
\end{document}